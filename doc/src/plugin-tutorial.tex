% DMTCP-Internal plugins:  make sure restart/resume operates after event:
%   RESTART/RESUME/REFILL/REGISTER\_NAME\_SERVICE\_DATA/SEND\_QUERIES
% Do we install distro packages to use /usr/include/dmtcp/plugin.h for compatibility?
% For KVM plugin, change MTCP break issue to be warning (not error)
% KVM plugin:  QEMU complains about negative value in signalfd

% TODO:  --with-icc
% TODO:  --vnc  (for dmtcp_launch)
% TODO:  test suite from Joshua Louie (dlopen prior to main; epoll taking too long)

\documentclass{article}
\usepackage{fullpage}
\usepackage[hidelinks]{hyperref}

\title{Tutorial for DMTCP Plugins}
\author{}
\date{March, 2015}

\begin{document}

\maketitle
\tableofcontents

\section{Introduction}

Plugins enable one to modify the behavior of DMTCP.  Two of the most
common uses of plugins are:
\begin{enumerate}
\item to execute an additional action at the time of checkpoint, resume,
	or restart.
\item to add a wrapper function around a call to a library function (including
	wrappers around system calls).
\end{enumerate}

Plugins are used for a variety of purposes. The {\tt DMTCP\_ROOT/contrib}
directory contains packages that users and developers have contributed
to be optionally loaded into DMTCP.


% POSSIBLE LIST:
% Torque, Condor, KVM/QEMU, record-replay, logging, restart on modified
% filesystem/new host/modified uid/etc., integration w/ Python for
% Python to call ckpt, ckpt over network to remote file

Plugin code is expressive, while requiring only a modest number of lines
of code.  The plugins in the contrib directory vary in size from
400 lines to 3000 lines of code.

Beginning with DMTCP version~2.0, much of DMTCP itself is also now
a plugin.  In this new design, the core DMTCP code is responsible
primarily for copying all of user space memory to a checkpoint
image file.  The remaining functions of DMTCP are handled by plugins,
found in {\tt DMTCP\_ROOT/plugin}.  Each plugin abstracts the essentials of a
different subsystem of the operating system and modifies its behavior
to accommodate checkpoint and restart.  Some of the subsystems for
which plugins have been written are: virtualization of process
and thread ids; files(open/close/dup/fopen/fclose/mmap/pty);
events (eventfd/epoll/poll/inotify/signalfd);
System~V IPC constructs (shmget/semget/msgget); TCP/IP
sockets (socket/connect/bind/listen/accept); and timers
(timer\_create/clock\_gettime).  (The indicated system calls are examples
only and not all-inclusive.)

\section{Anatomy of a plugin}

A plugin modifies the behavior of either DMTCP or a target application,
through three primary mechanisms, plus virtualization of ids.
\begin{description}
\item[Wrapper functions:]  One declares a wrapper function with the same
	name as an existing library function (including system calls in
	the run-time library).  The wrapper function can execute some
	prolog code, pass control to the ``real'' function,
	and then execute some epilog code.  Several plugins can wrap
	the same function in a nested manner.  One can also omit
	passing control to the ``real'' function, in order to shadow
	that function with an alternate behavior.
\item[Events:]  It is frequently useful to execute additional code
	at the time of checkpoint, or resume, or restart.  Plugins
	provide hook functions	to be called during these three events
	and numerous other important events in the life of a process.
\item[Coordinated checkpoint of distributed processes:]  DMTCP transparently
	checkpoints distributed computations across many nodes.
	At the time of checkpoint or restart, it may be necessary to
	coordinate information among the distributed processes.  For example,
	at restart time, an internal plugin of DMTCP allows the newly
	re-created processes to ``talk'' to their peers to discover the
	new network addresses of their peers.  This is important since
	a distributed computation may be restarted on a different cluster
	than its original one.
\item[Virtualization of ids:]  Ids (process id, timer id, sysv ipc id, etc.)
	are assigned by the kernel, by a peer process, and by remote processes.
	Upon restart, the external agent may wish to assign a different
	id than the one assigned prior to checkpoint.  Techniques for
	virtualization of ids are described in
        Section~\ref{sec:virtualization}.
\end{description}

\section{Writing Plugins}

\subsection{Invoking a plugin}

Plugins are just dynamic run-time libraries (.so files).
\hfill\break
\medskip\noindent
  \hspace{0.3truein} {\tt gcc -shared -fPIC -IDMTCP\_ROOT/dmtcp/include
                         -o PLUGIN1.so PLUGIN1.c}
\medskip

They are invoked at the beginning of a DMTCP computation
as command-line options:
\hfill\break
\medskip\noindent
  \hspace{0.3truein} {\tt dmtcp\_launch --with-plugin PLUGIN1.so:PLUGIN2.so
			  myapp}
\medskip

Note that one can invoke multiple plugins as a colon-separated list.
One should either specify a full path for each plugin (each .so~library),
or else to define LD\_LIBRARY\_PATH to include your own plugin directory.

\subsection{The plugin mechanisms}

The mechanisms of plugins are most easily described through examples.
This tutorial will rely on the examples in {\tt DMTCP\_ROOT/test/plugin}.
To get a feeling for the plugins, one can ``cd'' into each of the
subdirectories and execute: ``{\tt make check}''.

\subsubsection{Plugin events}

For context, please scan the code of {\tt DMTCP\_ROOT/plugin/example/example.c}.
Executing ``{\tt make check}'' will demonstrate the intended behavior.
Plugin events are handled by including the function {\tt dmtcp\_event\_hook}.
When a DMTCP plugin event occurs, DMTCP will call the
function {\tt dmtcp\_event\_hook} for each plugin.
This function is required only if the plugin will handle plugin events.
See Appendix~A for further details.

{\tt
\begin{verbatim}
void dmtcp_event_hook(DmtcpEvent_t event, DmtcpEventData_t *data)
{
  switch (event) {
  case DMTCP_EVENT_CHECKPOINT:
    printf("\n*** The plugin is being called before checkpointing. ***\n");
    break;
  case DMTCP_EVENT_RESUME:
    printf("*** Resume: the application has now been checkpointed. ***\n");
    break;
  case DMTCP_EVENT_RESTART:
    printf("*** The application is now being restarted. ***\n");
    break;
  ...
  default:
    break;
  }
  DMTCP_NEXT_EVENT_HOOK(event, data);
}
\end{verbatim}
}


% Plugin events:
%
% *** The model for events:  When a DMTCP event occurs, DMTCP calls ROUTINE in
% each plugin, in the order that the plugins were loaded, offering each plugin
% a chance to handle the event.  If ROUTINE is not defined in a plugin, DMTCP
% skips calling that plugin.  When ROUTINE is called, it is given the unique
% event id, and a switch statement can decide whether to take any special action.
% If no action is taken, ROUTINE returns XXX, and the next plugin is offered
% a chance to handle the event.  If a plugin does handle the event, a typical
% user code fragment will:
% A.  optionally carry out any pre-processing steps
% B.  optionally ask DMTCP to invoke the next event handler
% C.  optionally carry out any post-processing steps
%
% If all three steps are invoked, this effectively creates a wrapper function
% around any later plugins that handle the same event.  If step B is omitted,
% no further plugins will be offered the opportunity to handle the event.

\subsubsection{Plugin wrapper functions}
\label{sec:wrappers}

In its simplest form, a wrapper function can be written as follows:

{\tt
\begin{verbatim}
unsigned int sleep(unsigned int seconds) {
  static unsigned int (*next_fnc)() = NULL; /* Same type signature as sleep */
  struct timeval oldtv, tv;
  gettimeofday(&oldtv, NULL);
  time_t secs = val.tv_sec;
  printf("sleep1: "); print_time(); printf(" ... ");
  unsigned int result = NEXT_FNC(sleep)(seconds);
  gettimeofday(&tv, NULL);
  printf("Time elapsed:  %f\n",
          (1e6*(val.tv_sec-oldval.tv_sec) + 1.0*(val.tv_usec-oldval.tv_usec)) / 1e6);
  print_time(); printf("\n");

  return result;
}
\end{verbatim}
}

In the above example, we could also shadow the standard ``sleep'' function
by our own implementation, if we omit the call to ``{NEXT\_FNC}''.

\noindent
To see a related example, try:
\hfill\break
\medskip\noindent
  \hspace{0.3truein} {\tt cd DMTCP\_ROOT/test/plugin/sleep1; make check}
\medskip

\noindent
Wrapper functions from distinct plugins can be nested.  For a nesting
of plugin sleep2 around sleep1, do:
\hfill\break
\medskip\noindent
  \hspace{0.3truein} {\tt cd DMTCP\_ROOT/test/plugin; make; cd sleep2; make check}
\medskip

If one adds a wrapper around a function from a library other than libc.so
(e.g., libglx.so), it is best to dynamically link to that additional library:
\hfill\break
\medskip\noindent
  \hspace{0.3truein} {\tt gcc ...  -o PLUGIN1.so PLUGIN1.c -lglx.so}
\medskip

\subsubsection{Plugin coordination among multiple or distributed processes}
\label{sec:publishSubscribe}

It is often the case that an external agent will assign a particular
initial id to your process, but later assign a different id on restart.
Each process must re-discover its peers at restart time, without knowing
the pre-checkpoint ids.

DMTCP provides a ``Publish/Subscribe'' feature to enable communication among
peer processes. Two plugin events allow user
plugins to discover peers and pass information among peers.
The two events are:  {\tt DMTCP\_EVENT\_REGISTER\_NAME\_SERVICE\_DATA}
and {\tt DMTCP\_EVENT\_SEND\_QUERIES}.  DMTCP guarantees to provide a global
barrier between the two events.

An example of how to use the Publish/Subscribe feature is contained in
{\tt DMTCP\_ROOT/test/plugin/example-db}~.  The explanation below is best
understood in conjunction with reading that example.

A plugin processing {\tt DMTCP\_EVENT\_REGISTER\_NAME\_SERVICE\_DATA} should invoke: \\
int dmtcp\_send\_key\_val\_pair\_to\_coordinator(const void *key,
                                                   size\_t key\_len, \
                                                   const void *val, \
                                                   size\_t val\_len).

A plugin processing {\tt DMTCP\_EVENT\_SEND\_QUERIES} should invoke: \\
int dmtcp\_send\_query\_to\_coordinator(const void *key, size\_t key\_len,
                                            void *val, size\_t *val\_len).

\subsubsection{Using plugins to virtualize ids and other names}
\label{sec:virtualization}

Often an id or name will change between checkpoint and restart.
For example, on restart, the real pid of a process will change
from its pid prior to checkpoint.  Some DMTCP internal plugins
maintain a translation table in order to translate
between a virtualized id passed to the user code and a real id
maintained inside the kernel.  The ability to maintain this
translation table can also be used within third-party plugins.
For an example of adding virtualization to a plugin, see the plugin
in {\tt src/plugin/timer}.

In some less common cases, it can happen that a virtualized id is passed
to a library function by the target application.  Yet, that same library
function may be passed a real id by a second function from within the
same library.  In these cases, it is the responsibility of the
plugin implementor to choose a scheme that allows the first library
function to distinguish whether its argument is a virtual id (passed
from the target application) or a real id (passed from within the same
library).


% \section{Caveats}
%
% CAVEATS:
% Does your plugin break normal DMTCP?  to test this, modify DMTCP, and copy your
% plugin into {\tt DMTCP\_ROOT/lib}, and then run 'make check' for DMTCP as usual.
%
% SHARED MEMORY REGIONS:
% If two or more processes share a memory region, then the plugin writer must
% be clear on whether DMTCP or the plugin has responsibility for restoring
% the shared memory region.  Currently, EXPLAIN ....
%
% Virtualizing long-lived objects:  HOWTO
%
% INTERACTION OF MULTIPLE PLUGINS:
%     For simple plugins, this issue can be ignored.  But if your plugin has
% tables with long-lived data, other wrappers may create additional
% instantiations.  % It is reasonable for them to do this for temporary data
% structures at % the time of checkpoint or at the time of restart.  But
% normally, such an % object, created when the checkpoint event occurs, should
% be destroyed before % creating the actual checkpoint image.  Similarly, at
% restart time, if new % instances are created, they should be destroyed before
% returning control to the user threads.
%     It is polite for a plugin to check the above restrictions.  If it is
% violated, the plugin should print a warning about this.  This will help
% others, who % accidentally create long-lived objects at checkpoint- or
% restart-time, without intending to.  If the other plugin intends this unusual
% behavior, one can add a whitelist feature for other plugings to declare such
% intentions.
%     This small effort will provide a well-defined protocol that limits the
% interaction between distinct plugins.  Your effort helps others to debug their
% plugins, and a similar effort on their part will help you to debug your own
% plugin.
%
% Putting a printf inside a plugin at the time of checkpoint is dangerous.  This
% is because printf indirectly invokes a lock to prevent two threads from
% printing simultaneously.  This causes the checkpoint thread to call a printf.
% {\bf See README in test/plugin.}
%
% At checkpoint time, the DMTCP user thread will stop on that same lock.  This
% causes the two threads to deadlock.
%
% This use of conflicting locks is a bug in DMTCP as of DMTCP-2.0.  It will
% be fixed in the future version of DMTCP.


\section{Application-Initiated Checkpoints}

Application-initiated checkpoints are even simpler than full-featured
plugins.  In the simplest form, the following code can be executed both
with {\tt dmtcp\_launch} and without.:

{\tt
\begin{verbatim}
#include <stdio.h>
#include "dmtcp.h"
int main() {
  if (dmtcpCheckpoint() == DMTCP_NOT_PRESENT) {
    printf("dmtcpcheckpoint:  DMTCP not present.  No checkpoint is taken.\n");
  }
  return 0;
}
\end{verbatim}
}

For this program to be aware of DMTCP, it must be compiled with -fPIC and -ldl~:
\hfill\break
\medskip\noindent
  \hspace{0.3truein} {\tt gcc -fPIC -IDMTCP\_ROOT/dmtcp/include -o myapp
                      myapp.c -ldl}
\medskip

The most useful functions are:
\begin{list}{}{\leftmargin=3em \itemindent=-2em}
\item
  {\tt int dmtcp\_is\_enabled()} --- returns 1 when running with DMTCP;
                                     0 otherwise.
\item
  {\tt int dmtcp\_checkpoint()} --- returns DMTCP\_AFTER\_CHECKPOINT,
                                 DMTCP\_AFTER\_RESTART, or DMTCP\_NOT\_PRESENT.
\item
  {\tt int dmtcp\_disable\_ckpt()} --- DMTCP will block any
				 checkpoint requests.
\item
  {\tt int dmtcp\_enable\_ckpt()} --- DMTCP will execute any blocked
                checkpoint requests, and will permit new checkpoint requests.
\end{list}

The last two functions follow the common pattern of returning 0 on success
and {\tt DMTCP\_NOT\_PRESENT} if DMTCP is not present.
See the subdirectories
{\tt DMTCP\_ROOT/test/plugin/applic-initiated-ckpt}
and
{\tt DMTCP\_ROOT/test/plugin/applic-delayed-ckpt}, where one can execute
{\tt make check} for a live demonstration.

For advanced users, the file {\tt DMTCP\_ROOT/dmtcp/include/dmtcp.h}
also defines some additional functions not documented here:\\
~~{\tt dmtcp\_install\_hooks}\\
~~{\tt dmtcp\_get\_coordinator\_status}\\
~~{\tt dmtcp\_get\_local\_status}\\
~~{\tt dmtcp\_get\_uniquepid\_str}\\
~~{\tt dmtcp\_get\_ckpt\_filename}\\
and many others in {\tt dmtcp.h}.

\appendix
\section{Appendix:  Plugin Manual}
\subsection{Plugin events}
% \addcontentsline{toc}{chapter}{Appendix A:  Plugin Events}

\subsubsection{dmtcp\_event\_hook}

In order to handle DMTCP plugin events, a plugin must
define {\tt dmtcp\_event\_hook}.

\begin{verbatim}
NAME
       dmtcp_event_hook - Handle plugin events for this plugin

SYNOPSIS
       #include "dmtcp/plugin.h"

       void dmtcp_event_hook(DmtcpEvent_t event, DmtcpEventData_t *data)

DESCRIPTION
       When a plugin event occurs, DMTCP will look for the symbol
       dmtcp_event_hook in each plugin library.  If the symbol is found,
       that function will be called for the given plugin library.  DMTCP
       guarantees only to invoke the first such plugin library found in
       library search order.  Occurrences of dmtcp_event_hook in later
       plugin libraries will be called only if each previous function
       had invoked DMTCP_NEXT_EVENT_HOOK.  The argument, <event>, will be
       bound to the event being declared by DMTCP.  The argument, <data>,
       is required only for certain events.  See the following section,
       ``Plugin Events'' for a list of all events.

SEE ALSO
       DMTCP_NEXT_EVENT_HOOK
\end{verbatim}

\subsubsection{DMTCP\_NEXT\_EVENT\_HOOK}

A typical definition of {\tt dmtcp\_event\_hook} will invoke
{\tt DMTCP\_NEXT\_EVENT\_HOOK}.

\begin{verbatim}
NAME
       DMTCP_NEXT_EVENT_HOOK - call dmtcp_event_hook in next plugin library

SYNOPSIS
       #include "dmtcp/plugin.h"

       void DMTCP_NEXT_EVENT_HOOK(event, data)

DESCRIPTION
       This function must be invoked from within a plugin function library
       called dmtcp_event_hook.  The arguments <event> and <data> should
       normally be the same arguments passed to dmtcp_event_hook.

       DMTCP_NEXT_EVENT_HOOK may be called zero or one times.  If invoked zero
       times, no further plugin libraries will be called to handle events.
       The behavior is undefined  if DMTCP_NEXT_EVENT_HOOK is invoked more than
       once.  The typical usage of this function is to create a wrapper around
       the handling of the same event by later plugins.

SEE ALSO
       dmtcp_event_hook
\end{verbatim}

There are examples of compiling a plugin in the examples in {\tt
DMTCP\_ROOT/test/plugin}.

\subsubsection{Event Names}

The rest of this section defines plugin events.
The complete list of plugin events is always contained in
{\tt DMTCP\_ROOT/dmtcp/include/dmtcp.h}~.

DMTCP guarantees to call the dmtcp\_event\_hook function of the plugin
when the specified event occurs.
% If an event is thread-specific, DMTCP guarantees to call
% dmtcp\_event\_hook within the same thread.

Plugins that pass significant data through the data parameter
are marked with an asterisk: {}$^*$.
Most plugin events do not pass data through the data parameter.

Note that the events \\
   RESTART / RESUME / REFILL / REGISTER\_NAME\_SERVICE\_DATA / SEND\_QUERIES \\
should all be processed after the call to DMTCP\_NEXT\_EVENT\_HOOK() in
order to guarantee that the internal DMTCP plugins have first restored full
functionality.

\itemsep0pt
\subsubsection*{Checkpoint-Restart}
\begin{list}{}{\leftmargin=3em \itemindent=-2em}
\item
  DMTCP\_EVENT\_WRITE\_CKPT --- Invoked at final barrier before writing checkpoint
\item
  DMTCP\_EVENT\_RESTART --- Invoked at first barrier during restart of new process
\item
  DMTCP\_EVENT\_RESUME --- Invoked at first barrier during resume following checkpoint
\end{list}

\subsubsection*{Coordination of Multiple or Distributed Processes during Restart
	(see Appendix A.2.~Publish/Subscribe)}
\begin{list}{}{\leftmargin=3em \itemindent=-2em}
\item
  DMTCP\_EVENT\_REGISTER\_NAME\_SERVICE\_DATA$^*$ restart/resume
\item
  DMTCP\_EVENT\_SEND\_QUERIES$^*$ restart/resume
\end{list}

\subsubsection*{WARNING:  EXPERTS ONLY FOR REMAINING EVENTS}
\subsubsection*{Init/Fork/Exec/Exit}
\begin{list}{}{\leftmargin=3em \itemindent=-2em}
\item
  DMTCP\_EVENT\_INIT --- Invoked before main (in both the original program
and any new program called via exec)
\item
  DMTCP\_EVENT\_EXIT --- Invoked on call to exit/\_exit/\_Exit {\bf return from main?};
\item
  DMTCP\_EVENT\_PRE\_EXEC --- Invoked prior to call to exec
\item
  DMTCP\_EVENT\_POST\_EXEC --- Invoked before DMTCP\_EVENT\_INIT in new program
\item
  DMTCP\_EVENT\_ATFORK\_PREPARE --- Invoked before fork (see POSIX pthread\_atfork)
\item
  DMTCP\_EVENT\_ATFORK\_PARENT --- Invoked after fork by parent (see POSIX
    pthread\_atfork)
\item
  DMTCP\_EVENT\_ATFORK\_CHILD --- Invoked after fork by child (see POSIX
    pthread\_atfork) \\
\end{list}

\subsubsection*{Barriers (finer-grained control during checkpoint-restart)}
\begin{list}{}{\leftmargin=3em \itemindent=-2em}
\item
  DMTCP\_EVENT\_WAIT\_FOR\_SUSPEND\_MSG --- Invoked at barrier during
coordinated checkpoint
\item
  DMTCP\_EVENT\_THREADS\_SUSPEND --- Invoked at barrier during coordinated checkpoint
\item
  DMTCP\_EVENT\_LEADER\_ELECTION --- Invoked at barrier during coordinated checkpoint
\item
  DMTCP\_EVENT\_DRAIN --- Invoked at barrier during coordinated checkpoint
\item
  DMTCP\_EVENT\_WRITE\_CKPT --- Invoked at final barrier before writing checkpoint
\item
  DMTCP\_EVENT\_REFILL --- Invoked at first barrier during resume/restart of new process
\end{list}

\subsubsection*{Threads}
\begin{list}{}{\leftmargin=3em \itemindent=-2em}
\item
  DMTCP\_EVENT\_THREADS\_SUSPEND --- Invoked within checkpoint thread
	when all user threads have been suspended
\item
  DMTCP\_EVENT\_THREADS\_RESUME --- Invoked within checkpoint thread before
	any user threads are resumed. \\
	For debugging, consider calling the following code for this
        event:  {\tt static int x = 1; while(x);}
\item
  DMTCP\_EVENT\_PRE\_SUSPEND\_USER\_THREAD --- Each user thread invokes this prior
	to being suspended for a checkpoint
\item
  DMTCP\_EVENT\_RESUME\_USER\_THREAD --- Each user thread invokes this immediately
	after a resume or restart ({\tt isRestart()} available to plugin)
% \item
%   {\bf should we have separate DMTCP\_EVENT\_RESTART\_USER\_THREAD?
%     --- I vote yes.}
\item
  DMTCP\_EVENT\_THREAD\_START --- Invoked before start function given by clone
\item
  DMTCP\_EVENT\_THREAD\_CREATED --- Invoked within parent thread when clone call returns  (like parent for fork)
\item
  DMTCP\_EVENT\_PTHREAD\_START --- Invoked before start function given by pthread\_created
\item
  DMTCP\_EVENT\_PTHREAD\_EXIT --- Invoked before call to pthread\_exit
\item
  DMTCP\_EVENT\_PTHREAD\_RETURN --- Invoked in child thread when thread start function of pthread\_create returns
\item
\end{list}

\subsection{Publish/Subscribe}
% \addcontentsline({toc}{chapter}{Appendix B:  Publish/Subscribe}

Section~\ref{sec:publishSubscribe} provides an explanation of the
Publish/Subscribe feature for coordination among peer processes at resume-
or restart-time.  An example of how to use the Publish/Subscribe feature
is contained in {\tt DMTCP\_ROOT/test/plugin/example-db}~.

The primary events and functions used in this feature are:

\noindent
{\tt DMTCP\_EVENT\_REGISTER\_NAME\_SERVICE\_DATA} \\
\hbox{\ \ }
int dmtcp\_send\_key\_val\_pair\_to\_coordinator(const void *key,
                                                   size\_t key\_len,
                                                   const void *val,
                                                   size\_t val\_len) \\
{\tt DMTCP\_EVENT\_SEND\_QUERIES} \\
\hbox{\ \ }
int dmtcp\_send\_query\_to\_coordinator(const void *key, size\_t key\_len,
                                            void *val, size\_t *val\_len)

\subsection{Wrapper functions}
% \addcontentsline{toc}{chapter}{Appendix C:  Wrapper Functions}

For a description of including wrapper functions in a plugin, see
Section~\ref{sec:wrappers}.

\subsection{Miscellaneous utility functions}

Numerous DMTCP utility functions are provided that can be called from within
dmtcp\_event\_hook().  For a complete list, see
{\tt DMTCP\_ROOT/dmtcp/include/dmtcp/plugin.h}~.
The utility functions are still under active development, and may change
in small ways.  Some of the more commonly used utility functions follow.
Functions that return ``char *'' will not allocate memory, but instead will
return a pointer to a canonical string, which should not be changed.
{\tt
\begin{verbatim}
void dmtcp_get_local_ip_addr(struct in_addr *in);
const char* dmtcp_get_tmpdir(); /* given by --tmpdir, or DMTCP_TMPDIR, or TMPDIR */
const char* dmtcp_get_ckpt_dir();
   /* given by --ckptdir, or DMTCP_CHECKPOINT_DIR, or curr dir at ckpt time */
const char* dmtcp_get_ckpt_files_subdir();
int  dmtcp_get_ckpt_signal(); /* given by --ckpt-signal */
const char* dmtcp_get_uniquepid_str();
const char* dmtcp_get_computation_id_str();
uint64_t dmtcp_get_coordinator_timestamp();
uint32_t dmtcp_get_generation(); /* number of ckpt/restart sequences encountered */
const char* dmtcp_get_executable_path();
int dmtcp_get_restart_env(char *name, char *value, int maxvaluelen);
  /* For 'name' in environment, copy its value into 'value' param, but with
   * at most length 'maxvaluelen'.
   * Return 0 for success, and return code for various errors
   * See contrib/modify-env for an example of its use.
   */
\end{verbatim}
}


\end{document}
